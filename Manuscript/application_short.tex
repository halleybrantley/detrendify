\documentclass[12pt]{article}
\RequirePackage{amsthm,amsmath,amsbsy,amsfonts}
\usepackage{graphicx}
%\usepackage{enumerate}
\usepackage{natbib}
\usepackage{url} % not crucial - just used below for the URL 
\usepackage{placeins}

%\pdfminorversion=4
% NOTE: To produce blinded version, replace "0" with "1" below.
\newcommand{\blind}{0}

% DON'T change margins - should be 1 inch all around.
\addtolength{\oddsidemargin}{-.5in}%
\addtolength{\evensidemargin}{-.5in}%
\addtolength{\textwidth}{1in}%
\addtolength{\textheight}{1.3in}%
\addtolength{\topmargin}{-.8in}%


\begin{document}
	
	\section{Method comparison application}
	\begin{figure}[!h]
		\caption{Rugplot showing locations of signal after baseline removal using detrendr estimate of 15th quantile and a threshold of the median plus 3MAD.}
		\includegraphics[width = \linewidth]{Figures/corrected_rugplot.png}
	\end{figure}

	\begin{figure}[!h]
		\centering
		\caption{Variation of Information (VI) between sensor nodes after trend removal by quantile and method. The rows in the figure represent the three thresholds used: 3MAD, 4MAD, and 5MAD. Lower VI represents more similar classifications. }
		\includegraphics[width = .9\linewidth]{Figures/VI_app_short.png}
	\end{figure}
	
		\begin{figure}[!h]
		\centering
		\caption{Normalized Mutual Information (NMI) between sensor nodes after trend removal by quantile and method. The rows in the figure represent the three thresholds used: 3MAD, 4MAD, and 5MAD. Higher NMI represents more similar classifications. }
		\includegraphics[width = .9\linewidth]{Figures/NMI_app_short.png}
	\end{figure}

	\newpage
	%latex.default(confusion %>% filter(crit == 3) %>% select(-crit),     file = "../Manuscript/short_confusion_detrend_MAD3.tex",     rowlabel = "", rowname = "", colheads = c("Method", "Quantile",         "0,0,0", "1,0,0", "0,1,0", "1,1,0", "1,0,0", "1,1,0",         "1,0,1", "1,1,1"), caption = "Confusion matrices for 3 SPod nodes after baseline \n      removal (n=5000). Node order is f, g, h. The threshold for the signal was \n      set as the median + 3*MAD.")%
\begin{table}[!tbp]
\caption{Confusion matrices for 3 SPod nodes after baseline 
      removal (n=5000). Node order is f, g, h. The threshold for the signal was 
      set as the median + 3*MAD.\label{confusion}} 
\begin{center}
\begin{tabular}{llrrrrrrrrr}
\hline\hline
\multicolumn{1}{l}{}&\multicolumn{1}{c}{Method}&\multicolumn{1}{c}{Quantile}&\multicolumn{1}{c}{0,0,0}&\multicolumn{1}{c}{1,0,0}&\multicolumn{1}{c}{0,1,0}&\multicolumn{1}{c}{1,1,0}&\multicolumn{1}{c}{1,0,0}&\multicolumn{1}{c}{1,1,0}&\multicolumn{1}{c}{1,0,1}&\multicolumn{1}{c}{1,1,1}\tabularnewline
\hline
&detrendr&$0.05$&$4565$&$22$&$36$&$106$&$14$&$16$&$ 6$&$235$\tabularnewline
&qsreg&$0.05$&$4571$&$33$&$27$&$119$&$11$&$13$&$10$&$216$\tabularnewline
&detrendr&$0.10$&$4567$&$20$&$43$&$105$&$14$&$16$&$ 6$&$229$\tabularnewline
&qsreg&$0.10$&$4574$&$33$&$30$&$125$&$11$&$ 6$&$14$&$207$\tabularnewline
&detrendr&$0.15$&$4575$&$20$&$29$&$111$&$15$&$15$&$ 6$&$229$\tabularnewline
&qsreg&$0.15$&$4576$&$35$&$29$&$120$&$10$&$ 6$&$22$&$202$\tabularnewline
\hline
\end{tabular}\end{center}
\end{table}
 
	%latex.default(confusion %>% filter(crit == 4) %>% select(-crit),     file = "../Manuscript/short_confusion_detrend_MAD4.tex",     rowlabel = "", rowname = "", colheads = c("Method", "Quantile",         "0,0,0", "1,0,0", "0,1,0", "1,1,0", "1,0,0", "1,1,0",         "1,0,1", "1,1,1"), caption = "Confusion matrices for 3 SPod nodes after baseline \n      removal (n=6000). Node order is c, d, e. The threshold for the signal was \n      set as the median + 4*MAD.")%
\begin{table}[!tbp]
\caption{Confusion matrices for 3 SPod nodes after baseline 
      removal (n=6000). Node order is c, d, e. The threshold for the signal was 
      set as the median + 4*MAD.\label{confusion}} 
\begin{center}
\begin{tabular}{llrrrrrrrrr}
\hline\hline
\multicolumn{1}{l}{}&\multicolumn{1}{c}{Method}&\multicolumn{1}{c}{Quantile}&\multicolumn{1}{c}{0,0,0}&\multicolumn{1}{c}{1,0,0}&\multicolumn{1}{c}{0,1,0}&\multicolumn{1}{c}{1,1,0}&\multicolumn{1}{c}{1,0,0}&\multicolumn{1}{c}{1,1,0}&\multicolumn{1}{c}{1,0,1}&\multicolumn{1}{c}{1,1,1}\tabularnewline
\hline
&detrendr&$0.10$&$5632$&$53$&$  7$&$1$&$67$&$35$&$ 59$&$146$\tabularnewline
&qsreg&$0.10$&$5204$&$33$&$237$&$2$&$96$&$45$&$178$&$205$\tabularnewline
&detrendr&$0.15$&$5624$&$37$&$  3$&$1$&$85$&$48$&$ 60$&$142$\tabularnewline
&qsreg&$0.15$&$5273$&$29$&$141$&$2$&$94$&$41$&$210$&$210$\tabularnewline
\hline
\end{tabular}\end{center}
\end{table}
 
	%latex.default(confusion %>% filter(crit == 5) %>% select(-crit),     file = "../Manuscript/short_confusion_detrend_MAD5.tex",     rowlabel = "", rowname = "", colheads = c("Method", "Quantile",         "0,0,0", "1,0,0", "0,1,0", "1,1,0", "1,0,0", "1,1,0",         "1,0,1", "1,1,1"), caption = "Confusion matrices for 3 SPod nodes after baseline \n      removal (n=5000). Node order is f, g, h. The threshold for the signal was \n      set as the median + 5*MAD.")%
\begin{table}[!tbp]
\caption{Confusion matrices for 3 SPod nodes after baseline 
      removal (n=5000). Node order is f, g, h. The threshold for the signal was 
      set as the median + 5*MAD.\label{confusion}} 
\begin{center}
\begin{tabular}{llrrrrrrrrr}
\hline\hline
\multicolumn{1}{l}{}&\multicolumn{1}{c}{Method}&\multicolumn{1}{c}{Quantile}&\multicolumn{1}{c}{0,0,0}&\multicolumn{1}{c}{1,0,0}&\multicolumn{1}{c}{0,1,0}&\multicolumn{1}{c}{1,1,0}&\multicolumn{1}{c}{1,0,0}&\multicolumn{1}{c}{1,1,0}&\multicolumn{1}{c}{1,0,1}&\multicolumn{1}{c}{1,1,1}\tabularnewline
\hline
&detrendr&$0.05$&$4743$&$25$&$14$&$ 4$&$ 8$&$11$&$16$&$179$\tabularnewline
&qsreg&$0.05$&$4732$&$27$&$25$&$10$&$ 8$&$ 7$&$27$&$164$\tabularnewline
&detrendr&$0.10$&$4738$&$26$&$17$&$ 5$&$ 8$&$11$&$13$&$182$\tabularnewline
&qsreg&$0.10$&$4747$&$26$&$14$&$ 7$&$10$&$ 7$&$33$&$156$\tabularnewline
&detrendr&$0.15$&$4743$&$20$&$17$&$ 4$&$ 8$&$12$&$21$&$175$\tabularnewline
&qsreg&$0.15$&$4751$&$26$&$22$&$11$&$10$&$ 7$&$29$&$144$\tabularnewline
\hline
\end{tabular}\end{center}
\end{table}
 
	
\end{document}


